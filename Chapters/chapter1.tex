%!TEX root = ../template.tex
%%%%%%%%%%%%%%%%%%%%%%%%%%%%%%%%%%%%%%%%%%%%%%%%%%%%%%%%%%%%%%%%%%%
%% chapter1.tex
%% NOVA thesis document file
%%
%% Chapter with introduction
%%%%%%%%%%%%%%%%%%%%%%%%%%%%%%%%%%%%%%%%%%%%%%%%%%%%%%%%%%%%%%%%%%%

\typeout{NT FILE chapter1.tex}%

\chapter{Introduction}
\label{cha:introduction}

\section{Motivation}


Geographic Information Systems (GIS) have become indispensable tools in various disciplines, including urban planning, environmental management, public health, and transportation \cite{longley2015geographic}. These systems enable professionals to visualize, analyze, and interpret spatial data, facilitating informed decision-making based on geographic relationships and patterns. Despite the proliferation of GIS technologies and the increasing availability of geospatial data, the process of creating meaningful maps and conducting spatial analyses often remains confined to experts with specialized training. This expertise barrier limits the accessibility and utility of GIS for a broader audience, including policymakers, educators, and the general public \cite{haklay_neogeography_2013}.

The democratization of GIS has been an ongoing pursuit, with efforts focused on making spatial data and mapping tools more accessible to non-specialist users. Advancements in technology, such as web-based GIS platforms and open-source software, have lowered some barriers, allowing users to access spatial data and perform basic analyses without requiring extensive technical knowledge \cite{malakar_gis_2024}. However, these platforms often still necessitate familiarity with GIS concepts, terminologies, and interfaces, which can be intimidating for those without specialized training. This gap underscores the need for more intuitive and user-friendly interfaces that can cater to a diverse range of users with varying levels of expertise.

Simultaneously, the field of Artificial Intelligence (AI), particularly Natural Language Processing (NLP), has witnessed significant advancements. Large Language Models (LLM), such as OpenAI's GPT series, have demonstrated remarkable capabilities in understanding and generating human-like text \cite{b_evolution_2024}. These models can interpret nuanced queries, generate coherent responses, and perform tasks like code generation and translation. The integration of LLMs with GIS presents an opportunity to bridge the gap between expert GIS practitioners and non-specialist users. By enabling users to interact with GIS platforms through natural language queries, we can lower the barriers to entry, allowing a wider audience to leverage spatial data for various purposes.

Such integration could revolutionize how users access, analyze, and visualize geospatial information, making GIS more inclusive and user-friendly. For instance, policymakers could generate complex spatial analyses without needing to write queries in technical language or manipulate GIS software interfaces manually. Educators could create interactive maps for teaching purposes with simple verbal instructions, enhancing the learning experience for students. Additionally, the general public could engage with spatial data to better understand community planning initiatives, environmental changes, and public health trends.

However, this integration is not without challenges. Accurately interpreting natural language queries in a geospatial context requires the LLM to comprehend spatial concepts, relationships, and terminologies. Ensuring that the system retrieves the correct data and generates accurate maps is critical to maintaining trust and utility \cite{zhang_ethics_2023}. Ethical considerations, including data privacy, trustworthiness, and potential biases, must also be addressed to ensure responsible deployment. Misinterpretations of queries could lead to incorrect data retrieval and misleading visualizations, which could have serious implications in fields like public health or urban planning.

This thesis proposes the development of Prompt2Map, a system that leverages LLMs to convert natural language queries into interactive web maps. The system aims to facilitate geospatial data retrieval and map generation, making GIS more accessible to non-expert users. Through this work, we seek to contribute to ongoing efforts to democratize GIS and explore the intersection of AI and geospatial technologies. By addressing both technical and ethical challenges, Prompt2Map aspires to provide a reliable and user-friendly tool that empowers a diverse range of users to engage with spatial data effectively.

\section{Objectives}

The objectives of this thesis are threefold. First, to develop the Prompt2Map system, which is capable of converting natural language queries into interactive web maps by leveraging LLMs in conjunction with geospatial data retrieval techniques. The system should be able to interpret a wide range of queries, retrieve relevant data from various sources, and generate accurate and informative maps. This involves integrating advanced NLP capabilities with robust data processing and visualization tools to create a seamless user experience.

Second, to evaluate the effectiveness of Prompt2Map through a series of validation tests using synthetic prompts tailored to specific datasets. The evaluation will analyze metrics such as accuracy in data retrieval and map generation. This comprehensive evaluation aims to ensure that the system meets its performance goals and can handle diverse user needs and data complexities.

Third, to analyze the ethical implications associated with using LLMs for map generation. This includes issues of data privacy, potential biases in data and algorithms, alignment with user intent, risks of misinformation, and the potential for misuse. Ensuring ethical integrity is paramount to maintaining user trust and preventing harmful outcomes resulting from AI-generated maps.

By achieving these objectives, the thesis aims to contribute to making GIS tools more accessible and user-friendly, fostering greater engagement with spatial data among non-specialists, and promoting responsible use of AI in geospatial applications.

\section{Thesis Structure}

The thesis is organized into six chapters. The introduction provides the motivation for the study, outlines the objectives, and summarizes the thesis structure. The background chapter reviews the evolution of GIS towards greater accessibility, discusses the role of neogeography and self-service GIS platforms, introduces LLMs in the context of GIS, and addresses ethical considerations related to AI-generated maps.

The methods chapter details the system architecture of Prompt2Map, describes the datasets used for evaluation, and outlines the methodologies for system testing and ethical assessment. It delves into the technical aspects of how natural language queries are processed, data is retrieved, and maps are generated, providing a comprehensive overview of the system's components and workflows.

The results chapter presents the outcomes of validation tests, showcasing the system's ability to generate maps from natural language queries. It analyzes system performance in terms of accuracy, response time, and scalability, supported by various figures and tables that illustrate the findings. This chapter provides empirical evidence of the system's effectiveness and identifies areas where it excels or requires improvement.

The discussion interprets the findings, discusses the system's effectiveness in meeting its objectives, explores ethical implications, and identifies limitations and areas for improvement. It contextualizes the results within the broader landscape of GIS and AI, offering insights into how Prompt2Map contributes to the field and what future developments could enhance its capabilities.

Finally, the conclusions summarize the key findings, highlight the contributions to the fields of GIS and AI, discuss potential future work, and reflect on the broader impact of natural language interfaces in GIS. This chapter synthesizes the research, drawing connections between the objectives, methods, and results, and proposing directions for future exploration and innovation.

\section{Background}

\subsection{Democratization of GIS}

GIS have evolved significantly since their inception, transitioning from specialized tools used primarily by experts to more widely accessible platforms. Initially, GIS required substantial technical expertise, including knowledge of programming, database management, and spatial analysis techniques. The steep learning curve and high costs associated with GIS software limited its use to specific professional domains.

Advancements in technology and the proliferation of geospatial data have catalyzed the democratization of GIS. The development of user-friendly interfaces, open-source GIS software (e.g., QGIS), and web-based platforms (e.g., ArcGIS Online) has lowered barriers to entry. These tools have enabled a broader audience to engage with spatial data, perform basic analyses, and create maps without requiring extensive technical training. The increasing availability of high-quality geospatial data, coupled with more intuitive tools, has empowered non-specialist users to harness the power of GIS for various applications, from community planning to environmental monitoring.

The concept of neogeography emerged as individuals outside traditional geographic professions began using mapping tools for personal and community projects. Goodchild describes neogeography as the use of geographical techniques by non-expert users for non-professional purposes \cite{goodchild_neogeography_2009}. This movement has been facilitated by platforms like  OpenStreetMap, which allow users to contribute to and utilize geospatial information collaboratively. Neogeography has expanded the scope of GIS, enabling everyday users to participate in mapping activities, share spatial data, and contribute to collective knowledge bases. This participatory approach has democratized map-making, making it a communal and accessible activity.

Despite these advancements, a gap persists between the capabilities of expert GIS practitioners and those accessible to non-specialists. Complex spatial analyses, custom data queries, and advanced cartographic designs often remain beyond the reach of general users. This gap underscores the need for innovative approaches to further democratize GIS, making advanced functionalities more accessible and intuitive. Enhancing user interfaces, integrating intelligent assistants, and leveraging AI-driven tools are potential strategies to bridge this gap \cite{Frez2023252}, enabling a wider audience to perform sophisticated GIS tasks without specialized training.

\subsection{Self-Service GIS}

Self-service GIS platforms represent an important step toward making GIS more user-friendly. These platforms provide tools that simplify spatial data analysis and map creation through intuitive interfaces, often featuring drag-and-drop functionalities, guided workflows, and pre-configured templates \cite{rowland_towards_2020}. Users can perform tasks such as data visualization, basic spatial analysis, and thematic mapping without deep technical knowledge. By abstracting the complexity of GIS operations, self-service platforms enable users to focus on their specific analytical needs and visualization preferences.

However, self-service GIS platforms have limitations. They may not fully accommodate the diverse and complex needs of users, especially when dealing with sophisticated analyses or custom data queries. Users still need to understand GIS concepts, data structures, and the specific functionalities of the platform. The interfaces may not be flexible enough to handle unique or nuanced user requests. For instance, advanced spatial analyses like network analysis, spatial interpolation, or multi-criteria decision analysis often require specialized tools and a deeper understanding of GIS methodologies, which are not typically supported by self-service platforms.

To overcome these limitations, integrating natural language interfaces into GIS platforms has been proposed. Natural language interfaces allow users to interact with systems using everyday language, potentially reducing the learning curve and making GIS functionalities more accessible. By expressing queries and commands in natural language, users can bypass complex menus and technical commands. This approach leverages the advancements in NLP and AI to create more intuitive and responsive GIS tools that can cater to a wider range of user needs and expertise levels. Natural language interfaces can facilitate more dynamic and conversational interactions with GIS systems, enabling users to ask complex questions and receive immediate, contextually relevant responses.

\subsection{Large Language Models in GIS}

Large Language Models have revolutionized the field of NLP, offering new ways to interpret and generate human-like text. Models like GPT-4 have been trained on extensive datasets encompassing a wide range of topics, enabling them to understand context, disambiguate meanings, and generate coherent responses. These models excel at tasks such as language translation, text summarization, and question-answering, demonstrating a high level of proficiency in understanding and generating natural language.

In the context of GIS, LLMs can serve as intermediaries between the user and the system, interpreting natural language queries and translating them into actions that the GIS platform can execute. For example, a user could input a query like "Show me the areas in Lisbon with the highest unemployment rates," and the LLM would process this request, retrieve the relevant data, and generate the appropriate map. 

Furthermore, the concept of Retrieval-Augmented Generation (RAG) enhances LLM capabilities by combining retrieval mechanisms with generation. In a RAG system, the LLM retrieves relevant information from external data sources to inform its responses \cite{fan_survey_2024}. In GIS applications, RAG can enable the system to access geospatial databases dynamically, allowing for up-to-date and context-specific map generation. This approach ensures that the system can provide accurate and relevant visualizations based on the most current and pertinent data, enhancing the overall utility and reliability of the GIS platform.

Literature on the application of LLMs in GIS showcases various use cases where AI-driven natural language interfaces have significantly improved user interaction and data analysis capabilities. For instance, studies have explored how LLMs can assist in automated map generation and interactive spatial querying \cite{wang_gpt_2024}. These applications demonstrate the potential of LLMs to transform traditional GIS workflows, making spatial data analysis more efficient and accessible. Additionally, research has highlighted the benefits of integrating LLMs with GIS for educational purposes, allowing students to engage with spatial data through conversational interfaces and enhancing their learning experiences \cite{Mooney202385, Redican2024}.

Moreover, advancements in function calling and tool integration with LLMs have opened new avenues for enhancing GIS functionalities \cite{qu2024toolsurvey}. Function calling enables LLMs to execute specific functions or access external tools based on user queries, thereby extending the capabilities of the GIS platform beyond standard data retrieval and visualization tasks. This integration facilitates more sophisticated spatial analyses and dynamic map generation, catering to a broader range of user needs and enhancing the overall user experience.

\subsection{Retrieval-Augmented Generation}


Traditional RAG systems primarily rely on semantic proximity through embeddings to retrieve relevant documents or text snippets from large corpora. These systems encode both queries and documents into high-dimensional vector spaces, enabling the identification of semantically similar content based on their proximity in the embedding space \cite{gao_retrieval-augmented_2024}. This approach is effective for tasks such as question-answering and information retrieval, where the goal is to provide contextually relevant textual information to the user. 

In the context of geospatial data, RAG holds significant potential. Geospatial information is inherently dynamic, with frequent updates and changes in data due to various factors such as urban development, environmental changes, and demographic shifts. RAG allows GIS systems to access and incorporate the most recent data, ensuring that the generated maps and analyses reflect the current state of the spatial environment. This is crucial for applications that rely on real-time or near-real-time data, such as disaster management, traffic analysis, and environmental monitoring.

The state-of-the-art in RAG involves sophisticated retrieval algorithms and indexing techniques that enable efficient access to vast amounts of data. Advances in vector-based retrieval, where data is represented in high-dimensional vector spaces, have significantly improved the speed and accuracy of information retrieval in RAG systems \cite{gao_retrieval-augmented_2024}. Additionally, the integration of RAG with transformer-based architectures has enhanced the model's ability to understand and leverage retrieved information effectively, leading to more coherent and contextually appropriate responses.

\subsection{Text-to-SQL}

Structured Query Language (SQL) is a standardized programming language specifically designed for managing and manipulating relational databases. It provides a robust framework for performing a wide array of operations, including querying data, updating records, inserting new information, and deleting existing entries. SQL's declarative syntax allows users to specify what data they need without detailing the exact procedures to retrieve it, thereby simplifying complex data interactions. This language is fundamental for data analysts, database administrators, and applications that require efficient and precise data management. In the context of Geographic Information Systems, SQL is often extended with spatial functions that enable the querying and analysis of geospatial data, facilitating tasks such as spatial joins, proximity searches, and geographic aggregations.

%The task of translating natural language to database SQL queries is well-established in NLP, known as text-to-SQL. This task involves converting a natural language statement into a SQL query that can be executed against a database. Numerous benchmarks and datasets, such as Spider  and WikiSQL [@zhongSeq2SQL2017], have been developed to evaluate and improve models' performance on text-to-SQL tasks [@paperswithcode_text-to-sql]. These benchmarks provide standardized datasets and evaluation metrics that facilitate the training and assessment of models designed to perform text-to-SQL translation, ensuring that the models can handle a wide range of query complexities and database schemas.

The task of translating natural language to database SQL queries is well-established in NLP, known as text-to-SQL. This task involves converting a natural language statement into a SQL query that can be executed against a database. Modern text-to-SQL models leverage deep learning architectures, particularly transformer-based models, which excel at capturing the complex relationships between language and database schemas. These models are trained on large datasets that pair natural language questions with their corresponding SQL queries, enabling them to learn the patterns and structures necessary for accurate translation. Techniques such as schema linking, where the model identifies and aligns entities in the natural language query with database schema elements, have been instrumental in improving performance \cite{li2024pet}. In addition to handling simple queries, state-of-the-art text-to-SQL models are capable of managing complex queries involving multiple tables, nested subqueries, and advanced SQL functions. 

LLMs' ability to perform text-to-SQL translation can be harnessed in GIS to facilitate data retrieval. By integrating LLMs with GIS databases, users can query spatial data using natural language, and the system can generate the corresponding SQL queries to retrieve the data. This integration simplifies the data retrieval process, making it more accessible to users who may not have expertise in SQL or database management. Additionally, it allows for more dynamic and flexible querying capabilities, enabling users to perform complex spatial analyses through simple verbal instructions.

\subsection{Function Calling}

Function calling in LLMs refers to the capability of language models to invoke specific functions or access external tools based on the context of the input they receive. This feature extends the functionality of LLMs beyond text generation, allowing them to interact with external systems, perform computations, and access specialized tools dynamically. Function calling is a crucial advancement in the development of intelligent assistants, enabling more interactive and context-aware responses.

Function calling works by informing the model about the available external functions, including their names, descriptions, and expected parameters. When the model identifies a need to perform a specific task that aligns with an available function, it generates a function call with the appropriate arguments. This call is then executed by the system, and the results are incorporated into the model's response. The process ensures that the model's outputs adhere to the constraints and requirements of the external tools, enhancing the reliability and accuracy of the responses.

In the context of GIS, function calling enables the integration of specialized mapping and spatial analysis functions. For instance, when a user requests the generation of a choropleth map based on certain criteria, the LLM can invoke a predefined mapping function with the necessary parameters, such as data layers, color schemes, and attribute values. This automation streamlines the map creation process, allowing users to generate complex visualizations with simple natural language commands.

Function calling also facilitates the extension of the system's capabilities without necessitating continuous retraining of the LLM. As new functions are developed or existing ones are updated, they can be integrated into the system's function library, and the LLM can be instructed to utilize them as needed. This modular approach ensures that Prompt2Map remains adaptable and can incorporate advanced GIS functionalities seamlessly, catering to evolving user needs and technological advancements.

Moreover, function calling enhances the system's ability to maintain consistency and standardization in map generation. By relying on predefined functions with well-defined parameters, the system ensures that the generated maps adhere to specific cartographic principles and visualization standards. This consistency is crucial for producing reliable and professional-quality maps that meet user expectations and project requirements. Additionally, the use of predefined functions facilitates replicability, as the same functions can be invoked with identical parameters to reproduce maps under similar conditions. This replicability is essential for verifying results, conducting comparative analyses, and ensuring that map generation processes can be reliably repeated in future applications.

\subsection{Ethical Considerations}

The deployment of LLMs in GIS introduces a range of ethical considerations that are crucial to address for ensuring responsible and trustworthy use. Among these considerations are bias and fairness, data privacy, accuracy and the potential for misuse, as well as trustworthiness and reproducibility. Each of these factors plays a significant role in shaping the ethical landscape of AI-integrated GIS applications.

Bias and fairness stand out as prominent concerns in the integration of LLMs with GIS. LLMs are trained on extensive corpora of text, which inherently contain biases reflecting societal prejudices. These biases can inadvertently manifest in the model's outputs, leading to unfair or discriminatory results \cite{wang_aligning_2023, gallegos_bias_2024}. In the context of GIS, such biases may result in skewed data retrieval or the misrepresentation of specific areas or populations, thereby perpetuating existing inequalities. To mitigate these risks, it is essential to engage in careful dataset selection, implement robust bias detection and correction mechanisms, and maintain continuous monitoring of system outputs. These measures help ensure that the GIS outputs remain fair and unbiased, fostering equitable use across diverse user groups.

Data privacy is another critical ethical consideration, particularly because geospatial data often encompasses sensitive information, especially when dealing with demographic data at granular levels. Protecting user privacy and adhering to data protection regulations, such as the General Data Protection Regulation (GDPR), is paramount. GIS systems must incorporate measures to anonymize personal data, secure data storage and transmission, and prevent the inadvertent disclosure of sensitive information. Ensuring data privacy not only complies with legal standards but also builds user trust, which is essential for the widespread adoption of AI-driven GIS tools.

Accuracy and the potential for misuse present additional ethical challenges in the deployment of LLM-integrated GIS systems. LLMs have the capability to generate incorrect or misleading information, a phenomenon known as hallucinations, where the model produces plausible yet false outputs \cite{shuster_retrieval_2021}. In GIS applications, inaccuracies in data retrieval or map generation can lead to significant consequences, including incorrect conclusions or flawed decision-making. Furthermore, there is a risk that the system could be exploited to generate maps that infringe on privacy, compromise security, or spread misinformation. To address these issues, it is vital to implement validation checks, establish transparency measures, and enforce strict usage policies. These strategies help mitigate the risks associated with inaccuracies and misuse, ensuring that the GIS outputs are reliable and ethically sound.

Trustworthiness and reproducibility are essential for users to confidently rely on AI-generated maps. For GIS systems to be trustworthy, they must be transparent about how outputs are produced and allow for the reproducibility of results \cite{Li20234668, zhang_ethics_2023, wang_gpt_2024}. Providing users with access to information about data sources, processing methods, and inherent limitations enables them to understand and verify the results effectively. Reproducibility ensures that the same inputs will consistently yield the same outputs, which is critical for validating the reliability of GIS analyses and fostering user confidence in AI-driven tools.

Addressing these ethical considerations is imperative for the responsible deployment of LLM-integrated GIS systems. By proactively identifying potential risks and implementing comprehensive mitigation strategies, such systems can enhance trust, fairness, and reliability. Ensuring that GIS applications are ethically sound not only safeguards against misuse and biases but also promotes equitable access and utilization, ultimately contributing to more informed and just decision-making processes across various domains.