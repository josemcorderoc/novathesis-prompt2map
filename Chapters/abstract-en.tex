%!TEX root = ../template.tex
%%%%%%%%%%%%%%%%%%%%%%%%%%%%%%%%%%%%%%%%%%%%%%%%%%%%%%%%%%%%%%%%%%%%
%% abstract-en.tex
%% NOVA thesis document file
%%
%% Abstract in English([^%]*)
%%%%%%%%%%%%%%%%%%%%%%%%%%%%%%%%%%%%%%%%%%%%%%%%%%%%%%%%%%%%%%%%%%%%

\typeout{NT FILE abstract-en.tex}%

Geographic Information Systems (GIS) and web technologies have made map creation more accessible than ever before. However, a technical understanding of GIS tools is still required to produce cartographic outputs. 
This thesis introduces Prompt2Map, a system that converts natural language queries into web maps using Large Language Models (LLMs). 
Our approach is based on Retrieval-Augmented Generation (RAG), involving an initial step to extract data from geospatial sources, followed by a mapping step to visualize the data. 
Performance tests conducted over synthetic prompts demonstrate the system’s effectiveness. 
We also discuss ethical implications of this approach. 
This work contributes to bridging the gap between map consumers and producers, offering a natural language interface to authoritative geospatial data, thereby bringing spatial information closer to the general public.

\keywords{
  Large Language Models \and
  Geographic Information Systems \and
  Cartography \and
  Generative AI \and
  Retrieval-augmented generation
}