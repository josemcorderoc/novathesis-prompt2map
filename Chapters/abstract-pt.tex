%!TEX root = ../template.tex
%%%%%%%%%%%%%%%%%%%%%%%%%%%%%%%%%%%%%%%%%%%%%%%%%%%%%%%%%%%%%%%%%%%%
%% abstract-pt.tex
%% NOVA thesis document file
%%
%% Abstract in Portuguese
%%%%%%%%%%%%%%%%%%%%%%%%%%%%%%%%%%%%%%%%%%%%%%%%%%%%%%%%%%%%%%%%%%%%

\typeout{NT FILE abstract-pt.tex}%

Os Sistemas de Informação Geográfica (SIG) e as tecnologias web tornaram a criação de mapas mais acessível do que nunca. No entanto, ainda é necessário um conhecimento técnico das ferramentas de SIG para produzir resultados cartográficos.
Esta tese apresenta o Prompt2Map, um sistema que converte consultas em linguagem natural em mapas web utilizando Grandes Modelos de Linguagem (LLMs).
A nossa abordagem baseia-se em \textit{Retrieval-Augmented Generation} (RAG), envolvendo uma etapa inicial de extração de dados a partir de fontes geoespaciais, seguida de uma etapa de mapeamento para visualizar os dados.
Testes de desempenho realizados com consultas sintéticas demonstram a eficácia do sistema.
Discutimos também as implicações éticas desta abordagem.
Este trabalho contribui para reduzir a lacuna entre consumidores e produtores de mapas, oferecendo uma interface em linguagem natural para dados geoespaciais autoritativos, aproximando assim a informação espacial do público em geral.

\keywords{
  Grandes Modelos de Linguagem\and
  Sistemas de Informação Geográfica \and
  Cartografia \and
  IA Generativa \and
  RAG
}
