%!TEX root = ../template.tex
%%%%%%%%%%%%%%%%%%%%%%%%%%%%%%%%%%%%%%%%%%%%%%%%%%%%%%%%%%%%%%%%%%%%
%% chapter5.tex
%% NOVA thesis document file
%%
%% Chapter with lots of dummy text
%%%%%%%%%%%%%%%%%%%%%%%%%%%%%%%%%%%%%%%%%%%%%%%%%%%%%%%%%%%%%%%%%%%%

\typeout{NT FILE chapter5.tex}%

\chapter{Conclusion}
\label{cha:conclusion}

% max 2 pages

% 1. Recap of the Research
% TODO: Restate the research objectives, emphasizing the goal of making GIS tools accessible via natural language.
% TODO: Summarize the main contributions, including the development of Prompt2Map using LLMs and RAG.

% 2. Overall Findings
% TODO: Highlight the system's ability to generate web maps from natural language.
% TODO: Emphasize the significance of performance tests with synthetic prompts.
% TODO: Provide insights on bridging the gap between map consumers and producers.

% 3. Significance of the Research
% TODO: Discuss the academic impact of the research, particularly on AI and GIS fields.
% TODO: Explain its practical relevance for making geospatial data accessible to the public.
% TODO: Present a long-term vision for Prompt2Map's influence on future geospatial technology.

% 4. Limitations
% TODO: Describe the scope and methodology limitations, such as the use of synthetic prompts.
% TODO: Discuss challenges in data quality, LLM limitations, and scalability.

% 5. Future Directions
% TODO: Recommend specific improvements for Prompt2Map's functionality.
% TODO: Suggest new features or integrations, such as real-time mapping or complex dataset handling.
% TODO: Identify open research questions inspired by the thesis.

% 6. Final Thoughts
% TODO: Reflect on how Prompt2Map bridges the gap between map consumers and producers.
% TODO: Highlight aspirations for adoption and further research.
% TODO: Conclude with the significance of bringing spatial information closer to the public.

\section{Summary of Findings}

This thesis evaluated the capabilities of Prompt2Map, a Python package designed to convert natural language prompts into SQL queries for geospatial analysis, ultimately generating web maps. The system demonstrated high accuracy, reliability, and reproducibility across a range of test cases, with particularly strong performance on well-defined queries. The integration of natural language processing and geospatial data analysis in Prompt2Map represents a significant advancement in the accessibility of GIS tools, allowing users without technical expertise to interact intuitively with spatial data.

\section{Contribution}

Prompt2Map contributes to the growing body of research and applications at the intersection of GIS and NLP by lowering barriers to entry for geospatial analysis. Its reliance on SQL-based queries ensures reproducibility and traceability, distinguishing it from generative systems that often suffer from irreproducibility and potential inaccuracies. This focus on transparency and ethical design positions Prompt2Map as a trustworthy tool in geospatial science, aligning with recent calls for responsible AI development in cartography. Moreover, its potential for integrating Linked Data highlights its extensibility and future relevance in advancing data interoperability and comprehensive spatial analysis.

Addressing Ethical Challenges

The development and evaluation of Prompt2Map also addressed key ethical challenges associated with AI-generated maps, including risks of misleading information, unanticipated features, and reproducibility issues. Through rigorous testing and the adoption of transparent, SQL-based methodologies, the system minimizes these risks and ensures reliable outputs. Furthermore, its design supports the principles of equity and accessibility, enabling broader participation in spatial analysis.

\section{Future Directions}

Looking ahead, the integration of Linked Data into Prompt2Map presents an exciting opportunity to enhance its interoperability and analytical capabilities. Expanding its natural language understanding capabilities to handle more complex and ambiguous queries will further improve its usability and applicability. Additionally, conducting real-world user studies and incorporating privacy-preserving technologies will ensure the system remains robust, ethical, and user-centric as it evolves.

\section{Final remaks}

Prompt2Map exemplifies the transformative potential of combining natural language interfaces with geospatial analysis, paving the way for more inclusive and accessible GIS tools. By addressing critical challenges in accuracy, reproducibility, and ethics, the system provides a robust foundation for future advancements in geospatial science. This work underscores the importance of interdisciplinary approaches in developing innovative and equitable technologies, setting a standard for responsible and impactful applications of AI in GIS.