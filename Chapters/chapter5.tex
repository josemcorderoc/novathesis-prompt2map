%!TEX root = ../template.tex
%%%%%%%%%%%%%%%%%%%%%%%%%%%%%%%%%%%%%%%%%%%%%%%%%%%%%%%%%%%%%%%%%%%%
%% chapter5.tex
%% NOVA thesis document file
%%
%% Chapter with lots of dummy text
%%%%%%%%%%%%%%%%%%%%%%%%%%%%%%%%%%%%%%%%%%%%%%%%%%%%%%%%%%%%%%%%%%%%

\typeout{NT FILE chapter5.tex}%

\chapter{Conclusion}
\label{cha:conclusion}

% max 2 pages

% 1. Recap of the Research
% TODO: Restate the research objectives, emphasizing the goal of making GIS tools accessible via natural language.
% TODO: Summarize the main contributions, including the development of Prompt2Map using LLMs and RAG.

% 2. Overall Findings
% TODO: Highlight the system's ability to generate web maps from natural language.
% TODO: Emphasize the significance of performance tests with synthetic prompts.
% TODO: Provide insights on bridging the gap between map consumers and producers.

% 3. Significance of the Research
% TODO: Discuss the academic impact of the research, particularly on AI and GIS fields.
% TODO: Explain its practical relevance for making geospatial data accessible to the public.
% TODO: Present a long-term vision for Prompt2Map’s influence on future geospatial technology.

% 4. Limitations
% TODO: Describe the scope and methodology limitations, such as the use of synthetic prompts.
% TODO: Discuss challenges in data quality, LLM limitations, and scalability.

% 5. Future Directions
% TODO: Recommend specific improvements for Prompt2Map’s functionality.
% TODO: Suggest new features or integrations, such as real-time mapping or complex dataset handling.
% TODO: Identify open research questions inspired by the thesis.

% 6. Final Thoughts
% TODO: Reflect on how Prompt2Map bridges the gap between map consumers and producers.
% TODO: Highlight aspirations for adoption and further research.
% TODO: Conclude with the significance of bringing spatial information closer to the public.