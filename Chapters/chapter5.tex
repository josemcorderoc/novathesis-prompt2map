%!TEX root = ../template.tex
%%%%%%%%%%%%%%%%%%%%%%%%%%%%%%%%%%%%%%%%%%%%%%%%%%%%%%%%%%%%%%%%%%%%
%% chapter5.tex
%% NOVA thesis document file
%%
%% Chapter with lots of dummy text
%%%%%%%%%%%%%%%%%%%%%%%%%%%%%%%%%%%%%%%%%%%%%%%%%%%%%%%%%%%%%%%%%%%%

\typeout{NT FILE chapter5.tex}%
\chapter{Conclusions}
\label{cha:conclusions}

This thesis set out to address the challenges of making GIS more accessible to non-expert users by leveraging advancements in AI, specifically LLMs. The primary objective was to develop Prompt2Map, a system that allows users to interact with geospatial data and generate web maps through natural language queries. By integrating LLMs with RAG, Prompt2Map bridges the gap between traditional GIS workflows and intuitive, user-friendly interfaces. The main contributions of this research include the design and implementation of Prompt2Map, showcasing its capability to interpret natural language, retrieve relevant geospatial data, and generate accurate, reproducible maps. The system's development involved addressing technical challenges, such as natural language understanding, SQL generation, and geospatial visualization.

\section{Contributions}

This research makes significant contributions to both academic and practical domains. Academically, it advances the intersection of AI and GIS, demonstrating how LLMs can be effectively utilized to interpret spatial queries and automate map generation. The integration of RAG techniques further enriches the field, offering a framework for retrieving and utilizing real-time geospatial data in natural language-driven systems.

Practically, Prompt2Map underscores the importance of making geospatial tools accessible to a wider audience. By simplifying interactions with spatial data, the system empowers policymakers, educators, and the general public to engage with geospatial information for decision-making, education, and community initiatives. Performance tests with synthetic prompts validated the system's ability to handle diverse user queries reliably, providing a robust foundation for its applicability in real-world scenarios.

These findings highlight the transformative potential of natural language interfaces to bridge the gap between map producers—GIS experts—and map consumers, such as non-specialist users. By fostering more intuitive interactions with geospatial data, Prompt2Map paves the way for more inclusive and equitable access to spatial insights. In the long term, it has the potential to influence the design of future geospatial technologies, contributing to both research and practice.

\section{Limitations}

While this research demonstrates the viability of natural language-driven GIS tools, certain limitations remain. The use of synthetic prompts for evaluation, while effective for initial testing, may not fully capture the diversity of real-world user queries. The system's reliance on LLMs introduces challenges related to data quality, ambiguity in language interpretation, and scalability when handling complex datasets or queries.

Additionally, Prompt2Map's performance depends on the quality and structure of the underlying geospatial data. Issues such as outdated information or inconsistencies in data formats could impact the accuracy and reliability of the generated maps. Addressing these limitations is essential for enhancing the system's robustness and applicability in diverse contexts.

\section{Future Directions}

To improve Prompt2Map, several enhancements are recommended. First, incorporating real-time data retrieval capabilities would enable the system to generate maps reflecting current conditions, such as traffic patterns or weather events. Second, expanding the system's natural language understanding capabilities to handle more complex queries and ambiguous language constructs would improve user experience. Third, integrating advanced features, such as multi-layered map visualizations and interactive data exploration tools, would cater to a broader range of user needs.

Future research could explore the integration of Prompt2Map with external geospatial platforms like OpenStreetMap or ArcGIS, enabling richer analyses and data interoperability. Conducting user studies would provide insights into the system's usability and inform further refinements. Additionally, addressing ethical challenges, such as ensuring fairness and mitigating biases in LLM-generated outputs, remains a critical area for ongoing investigation.

\section{Final Thoughts}

Prompt2Map represents a significant step toward bridging the gap between map consumers and producers, making geospatial information more accessible and actionable. By enabling users to interact with GIS through natural language, the system lowers barriers to entry and fosters greater engagement with spatial data.

The aspirations for Prompt2Map extend beyond its current capabilities. By continuing to innovate and address challenges, this research hopes to inspire further exploration at the intersection of AI and GIS. Ultimately, Prompt2Map exemplifies the transformative potential of bringing spatial information closer to the public, empowering individuals and communities to make informed decisions based on geographic insights.