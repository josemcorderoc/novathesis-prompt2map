%!TEX root = ../template.tex
%%%%%%%%%%%%%%%%%%%%%%%%%%%%%%%%%%%%%%%%%%%%%%%%%%%%%%%%%%%%%%%%%%%%
%% chapter4.tex
%% NOVA thesis document file
%%
%% Chapter with lots of dummy text
%%%%%%%%%%%%%%%%%%%%%%%%%%%%%%%%%%%%%%%%%%%%%%%%%%%%%%%%%%%%%%%%%%%%

\typeout{NT FILE chapter4.tex}%

\chapter{Discussion}
\label{cha:discussion}

% 1. Introduction
% // TODO: Brief overview of the discussion's purpose.
% // TODO: Recap of thesis objectives, emphasizing the need to simplify map creation and make geospatial data accessible through natural language interfaces.

% 2. Key Findings
% // TODO: Summarize how Prompt2Map converts natural language into web maps using LLMs and RAG.
% // TODO: Highlight the system's ability to handle synthetic prompts effectively in performance tests.
% // TODO: Discuss the implications of successfully bridging the gap between map producers and consumers.
% // TODO: Evaluate the practicality of natural language interfaces in making GIS tools more inclusive.

% 3. Comparison with Existing Solutions
% // TODO: Fig Table comparing Prompt2Map with traditional GIS tools in terms of features.
% // TODO: Provide an overview of GIS tools requiring technical expertise versus Prompt2Map’s simplicity.
% // TODO: Discuss advantages of Prompt2Map, such as its ability to use RAG for data extraction and mapping.
% // TODO: Identify limitations in handling ambiguous or overly complex natural language queries.

% 4. Strengths and Limitations
% // TODO: Explain the effectiveness of RAG in integrating geospatial data extraction and visualization.
% // TODO: Highlight the system’s accessibility and usability for a broader audience.
% // TODO: Discuss limitations, such as the dependency on LLM quality and geospatial data precision.
% // TODO: Address potential scalability and performance issues.

% 5. Ethical Implications
% // TODO: Discuss how Prompt2Map ensures data privacy when handling sensitive geospatial information.
% // TODO: Address how biases in source data might propagate into maps.
% // TODO: Evaluate the responsibility of using such systems in critical applications like disaster response.

% 6. Broader Implications
% // TODO: Highlight practical applications in education, urban planning, and public information.
% // TODO: Describe how Prompt2Map advances research in GIS, LLMs, and human-computer interaction.
% // TODO: Discuss how the system democratizes GIS by bringing spatial information closer to the public.

% 7. Future Work
% // TODO: Propose enhancements in LLM performance and RAG workflows.
% // TODO: Suggest extending Prompt2Map to handle temporal or 3D geospatial data.
% // TODO: Explore how natural language interfaces could evolve for geospatial technologies.

% 8. Conclusion of the Discussion
% // TODO: Briefly summarize the discussion and transition to the conclusion chapter.
