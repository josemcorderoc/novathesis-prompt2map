%!TEX root = ../template.tex
%%%%%%%%%%%%%%%%%%%%%%%%%%%%%%%%%%%%%%%%%%%%%%%%%%%%%%%%%%%%%%%%%%%%
%% chapter4.tex
%% NOVA thesis document file
%%
%% Chapter with lots of dummy text
%%%%%%%%%%%%%%%%%%%%%%%%%%%%%%%%%%%%%%%%%%%%%%%%%%%%%%%%%%%%%%%%%%%%

\typeout{NT FILE chapter4.tex}%

\chapter{Discussion}
\label{cha:discussion}

\section{Interpretation of Results}

The evaluation of Prompt2Map demonstrates its capability to effectively translate natural language prompts into SQL queries for generating accurate and reproducible geospatial maps. The system achieved high completion rates and macro precision across most test cases, reflecting its reliability in handling well-defined prompts. The scatter matrix analysis revealed positive correlations among metrics such as macro F1-score, frequency of the mode, and consistency entropy. These correlations emphasize that consistent and accurate outputs often go hand-in-hand, with better performance in one area reinforcing reliability across others.

High-performing questions exemplified how clear language, direct schema mappings, and standard SQL constructs enable robust query generation. Conversely, low-performing questions exposed the system's limitations in handling ambiguous prompts or advanced SQL requirements, such as division by zero or multi-column computations. These challenges underscore the trade-offs between simplicity and versatility in natural language interfaces for GIS.

\section{Comparison with Existing Work}

\subsection{Implementation Status of Existing Solutions}

Prompt2Map exists within a broader context of natural language GIS tools, each with varying degrees of implementation and readiness. Systems like GeoGPT \cite{kim_flex_2024}, GeoQAMap \cite{feng_geoqamap_2023}, and Aino\footnote{\url{https://aino.world/}} represent different approaches to integrating LLMs with geospatial systems. GeoGPT is a research-oriented framework designed to handle complex workflows and automate GIS tasks, serving as a valuable proof of concept but lacking commercialization. GeoQAMap  operates as an experimental system capable of generating SPARQL queries and visualizing results based on geospatial knowledge bases like Wikidata. In contrast, Aino is a fully implemented and commercially available platform for geospatial data analysis and visualization.

\subsection{Comparison to Prompt2Map}

Prompt2Map shares similarities with these systems but also has unique strengths and limitations. Like GeoQAMap and Aino, it uses natural language as the primary interface, offering accessibility to non-expert users. Prompt2Map automates the mapping process in a manner similar to GeoQAMap's SPARQL generation and GeoGPT's GIS workflow automation, but it is more specialized, focusing specifically on generating maps rather than handling a broad spectrum of geospatial operations.

In terms of implementation, Prompt2Map is a functional, open-source Python package available on GitHub, making it more accessible than research prototypes like GeoGPT and GeoQAMap. However, it lacks the commercial polish and collaborative features of Aino, which positions it as a business-ready product. Additionally, Prompt2Map allows tailored map generation with specific parameters, aligning with the extensibility goals of GeoGPT and Aino. Yet, it does not yet match GeoGPT's integration with professional GIS tools and geoprocessing.

Prompt2Map occupies a promising middle ground between research prototypes and commercial solutions. With further development, it has the potential to bridge this gap by integrating external datasets, such as OpenStreetMap, or connecting to web-based GIS platforms. This would enhance its analytical capabilities and broaden its appeal to a wider audience.
\section{Addressing Ethical Challenges}

Prompt2Map effectively addresses several ethical concerns identified in the literature. Misleading information, a key risk in AI-generated maps \cite{zhang_ethics_2023}, can occur if SQL queries are incorrect or poorly generated. By incorporating rigorous testing and evaluation metrics, such as the macro F1-score, Prompt2Map reduces the likelihood of such errors. Unanticipated features, often indicative of low precision, are similarly mitigated through robust evaluation and continuous system improvements.

Reproducibility, a significant challenge for many generative systems, is a core strength of Prompt2Map. Each generated map is traceable to a specific SQL query and dataset, ensuring outputs are transparent and replicable. This aligns with calls for reproducibility in geospatial research \cite{zhang_ethics_2023}, further supporting the system's reliability. While Prompt2Map does not process sensitive individual-level data, it remains crucial to maintain robust data privacy measures, particularly if the system integrates more complex datasets in the future.

\section{Implications for GIS and NLP}

The integration of natural language interfaces into GIS systems, as exemplified by Prompt2Map, offers transformative potential. By lowering the barrier to entry for non-expert users, these systems democratize access to spatial analysis and visualization tools. This aligns with the broader vision of self-service GIS systems \cite{rowland_towards_2020}, and contributes to a growing body of work exploring the intersection of GIS and NLP technologies.

Prompt2Map's approach demonstrates how SQL-based systems can serve as a foundation for ethical and reproducible GIS tools. The potential to integrate Linked Data presents an exciting avenue for future work, enabling richer data interoperability and expanding the system's analytical capabilities. Such enhancements could position Prompt2Map as a key tool for advancing equity and accessibility in geospatial science \cite{wang_gpt_2024}. Moreover, the system's simplicity and usability make it well-suited for educational applications, urban planning, and public information dissemination.

\section{Contributions}

This research makes significant contributions to both academic and practical domains. Academically, it advances the intersection of AI and GIS, demonstrating how LLMs can be effectively utilized to interpret spatial queries and automate map generation. The integration of RAG techniques further enriches the field, offering a framework for retrieving and utilizing real-time geospatial data in natural language-driven systems.

Practically, Prompt2Map underscores the importance of making geospatial tools accessible to a wider audience. By simplifying interactions with spatial data, the system empowers policymakers, educators, and the general public to engage with geospatial information for decision-making, education, and community initiatives. Performance tests with synthetic prompts validated the system's ability to handle diverse user queries reliably, providing a robust foundation for its applicability in real-world scenarios.

These findings highlight the transformative potential of natural language interfaces to bridge the gap between map producers—GIS experts—and map consumers, such as non-specialist users. By fostering more intuitive interactions with geospatial data, Prompt2Map paves the way for more inclusive and equitable access to spatial insights. In the long term, it has the potential to influence the design of future geospatial technologies, contributing to both research and practice.

\section{Limitations}

While this research demonstrates the viability of natural language-driven GIS tools, certain limitations remain. The use of synthetic prompts for evaluation, while effective for initial testing, may not fully capture the diversity of real-world user queries. The system's reliance on LLMs introduces challenges related to data quality, ambiguity in language interpretation, and scalability when handling complex datasets or queries.

Additionally, Prompt2Map's performance depends on the quality and structure of the underlying geospatial data. Issues such as outdated information or inconsistencies in data formats could impact the accuracy and reliability of the generated maps. Addressing these limitations is essential for enhancing the system's robustness and applicability in diverse contexts.

\section{Future Directions}

To improve Prompt2Map, several enhancements are recommended. First, incorporating real-time data retrieval capabilities would enable the system to generate maps reflecting current conditions, such as traffic patterns or weather events. Second, expanding the system's natural language understanding capabilities to handle more complex queries and ambiguous language constructs would improve user experience. Third, integrating advanced features, such as multi-layered map visualizations and interactive data exploration tools, would cater to a broader range of user needs.

Future research could explore the integration of Prompt2Map with external geospatial platforms like OpenStreetMap or ArcGIS, enabling richer analyses and data interoperability. Conducting user studies would provide insights into the system's usability and inform further refinements. Additionally, addressing ethical challenges, such as ensuring fairness and mitigating biases in LLM-generated outputs, remains a critical area for ongoing investigation.