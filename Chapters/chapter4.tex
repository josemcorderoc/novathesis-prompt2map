%!TEX root = ../template.tex
%%%%%%%%%%%%%%%%%%%%%%%%%%%%%%%%%%%%%%%%%%%%%%%%%%%%%%%%%%%%%%%%%%%%
%% chapter4.tex
%% NOVA thesis document file
%%
%% Chapter with lots of dummy text
%%%%%%%%%%%%%%%%%%%%%%%%%%%%%%%%%%%%%%%%%%%%%%%%%%%%%%%%%%%%%%%%%%%%

\typeout{NT FILE chapter4.tex}%

\chapter{Discussion}
\label{cha:discussion}

% 1. Introduction
% // TODO: Brief overview of the discussion's purpose.
% // TODO: Recap of thesis objectives, emphasizing the need to simplify map creation and make geospatial data accessible through natural language interfaces.

% 2. Key Findings
% // TODO: Summarize how Prompt2Map converts natural language into web maps using LLMs and RAG.
% // TODO: Highlight the system's ability to handle synthetic prompts effectively in performance tests.
% // TODO: Discuss the implications of successfully bridging the gap between map producers and consumers.
% // TODO: Evaluate the practicality of natural language interfaces in making GIS tools more inclusive.

% 3. Comparison with Existing Solutions
% // TODO: Fig Table comparing Prompt2Map with traditional GIS tools in terms of features.
% // TODO: Provide an overview of GIS tools requiring technical expertise versus Prompt2Map's simplicity.
% // TODO: Discuss advantages of Prompt2Map, such as its ability to use RAG for data extraction and mapping.
% // TODO: Identify limitations in handling ambiguous or overly complex natural language queries.

% 4. Strengths and Limitations
% // TODO: Explain the effectiveness of RAG in integrating geospatial data extraction and visualization.
% // TODO: Highlight the system's accessibility and usability for a broader audience.
% // TODO: Discuss limitations, such as the dependency on LLM quality and geospatial data precision.
% // TODO: Address potential scalability and performance issues.

% 5. Ethical Implications
% // TODO: Discuss how Prompt2Map ensures data privacy when handling sensitive geospatial information.
% // TODO: Address how biases in source data might propagate into maps.
% // TODO: Evaluate the responsibility of using such systems in critical applications like disaster response.

% 6. Broader Implications
% // TODO: Highlight practical applications in education, urban planning, and public information.
% // TODO: Describe how Prompt2Map advances research in GIS, LLMs, and human-computer interaction.
% // TODO: Discuss how the system democratizes GIS by bringing spatial information closer to the public.

% 7. Future Work
% // TODO: Propose enhancements in LLM performance and RAG workflows.
% // TODO: Suggest extending Prompt2Map to handle temporal or 3D geospatial data.
% // TODO: Explore how natural language interfaces could evolve for geospatial technologies.

% 8. Conclusion of the Discussion
% // TODO: Briefly summarize the discussion and transition to the conclusion chapter.


\section{Interpretation of Results}

The evaluation of Prompt2Map demonstrates its capability to effectively translate natural language prompts into SQL queries for generating accurate and reproducible geospatial maps. The system achieved high completion rates and macro precision across most test cases, reflecting its reliability in handling well-defined prompts. The scatter matrix analysis revealed positive correlations among metrics such as macro F1-score, frequency of the mode, and consistency entropy. These correlations emphasize that consistent and accurate outputs often go hand-in-hand, with better performance in one area reinforcing reliability across others.

High-performing questions exemplified how clear language, direct schema mappings, and standard SQL constructs enable robust query generation. Conversely, low-performing questions exposed the system's limitations in handling ambiguous prompts or advanced SQL requirements, such as division by zero or multi-column computations. These challenges underscore the trade-offs between simplicity and versatility in natural language interfaces for GIS.

\section{Comparison with Existing Work}

\subsection{Implementation Status of Existing Solutions}

Prompt2Map exists within a broader context of natural language GIS tools, each with varying degrees of implementation and readiness. Systems like GeoGPT \cite{kim_flex_2024}, GeoQAMap \cite{feng_geoqamap_2023}, and Aino\footnote{\url{https://aino.world/}} represent different approaches to integrating LLMs with geospatial systems. GeoGPT is a research-oriented framework designed to handle complex workflows and automate GIS tasks, serving as a valuable proof of concept but lacking commercialization. GeoQAMap  operates as an experimental system capable of generating SPARQL queries and visualizing results based on geospatial knowledge bases like Wikidata. In contrast, Aino is a fully implemented and commercially available platform for geospatial data analysis and visualization.

\subsection{Comparison to Prompt2Map}

Prompt2Map shares similarities with these systems but also has unique strengths and limitations. Like GeoQAMap and Aino, it uses natural language as the primary interface, offering accessibility to non-expert users. Prompt2Map automates the mapping process in a manner similar to GeoQAMap's SPARQL generation and GeoGPT's GIS workflow automation, but it is more specialized, focusing specifically on generating maps rather than handling a broad spectrum of geospatial operations.

In terms of implementation, Prompt2Map is a functional, open-source Python package available on GitHub, making it more accessible than research prototypes like GeoGPT and GeoQAMap. However, it lacks the commercial polish and collaborative features of Aino, which positions it as a business-ready product. Additionally, Prompt2Map allows tailored map generation with specific parameters, aligning with the extensibility goals of GeoGPT and Aino. Yet, it does not yet match GeoGPT's integration with professional GIS tools and geoprocessing.

Prompt2Map occupies a promising middle ground between research prototypes and commercial solutions. With further development, it has the potential to bridge this gap by integrating external datasets, such as OpenStreetMap, or connecting to web-based GIS platforms. This would enhance its analytical capabilities and broaden its appeal to a wider audience.
\subsection{Addressing Ethical Challenges}

Prompt2Map effectively addresses several ethical concerns identified in the literature. Misleading information, a key risk in AI-generated maps \cite{zhang_ethics_2023}, can occur if SQL queries are incorrect or poorly generated. By incorporating rigorous testing and evaluation metrics, such as the macro F1-score, Prompt2Map reduces the likelihood of such errors. Unanticipated features, often indicative of low precision, are similarly mitigated through robust evaluation and continuous system improvements.

Reproducibility, a significant challenge for many generative systems, is a core strength of Prompt2Map. Each generated map is traceable to a specific SQL query and dataset, ensuring outputs are transparent and replicable. This aligns with calls for reproducibility in geospatial research \cite{zhang_ethics_2023}, further supporting the system's reliability. While Prompt2Map does not process sensitive individual-level data, it remains crucial to maintain robust data privacy measures, particularly if the system integrates more complex datasets in the future.

\section{Implications for GIS and NLP}

The integration of natural language interfaces into GIS systems, as exemplified by Prompt2Map, offers transformative potential. By lowering the barrier to entry for non-expert users, these systems democratize access to spatial analysis and visualization tools. This aligns with the broader vision of self-service GIS systems \cite{rowland_towards_2020}, and contributes to a growing body of work exploring the intersection of GIS and NLP technologies.

Prompt2Map's approach demonstrates how SQL-based systems can serve as a foundation for ethical and reproducible GIS tools. The potential to integrate Linked Data presents an exciting avenue for future work, enabling richer data interoperability and expanding the system's analytical capabilities. Such enhancements could position Prompt2Map as a key tool for advancing equity and accessibility in geospatial science \cite{wang_gpt_2024}. Moreover, the system's simplicity and usability make it well-suited for educational applications, urban planning, and public information dissemination.

% \subsection{Limitations and Future Work}

% Prompt2Map exhibits limitations in handling complex queries, ambiguous language constructs, and advanced SQL operations. Enhancing the system's natural language understanding capabilities and incorporating more sophisticated error-handling mechanisms will be critical to addressing these challenges. Future work could focus on expanding the system's performance in diverse real-world scenarios, conducting user studies to refine its usability, and incorporating privacy-preserving technologies to ensure robust data protection.

% The integration of Linked Data, while promising, would require careful consideration of interoperability and data governance issues. Exploring connections to external datasets such as OpenStreetMap or connecting to web-based GIS platforms could significantly enhance the system's capabilities. Additionally, ensuring compliance with data protection regulations will be essential as the system evolves.
